\documentclass{article}

\usepackage{algorithm}
\usepackage{algpseudocode}

\algnewcommand{\LineComment}[1]{\State \(\triangleright\) #1}

\begin{document}

  \begin{algorithm}
    \caption{Basic Rule in Handling RGBA Image}
    \label{RGB}
    \begin{algorithmic}[1]
      \Function{RGBA}{$Function$, $data$, $args$, $mode$}
      \LineComment apply $Function$ to $data$ with $args$
      \If{$mode$ in $'RGBA'$} \Comment RGBA
      \LineComment $data$ will look like $[R,G,B]$ or $[R,G,B,A]$ 
      \\ 
      \State $rgb$ = \Call{$Function$}{$data[:3]$, $args$, $mode$='L'} \Comment $L$ means 1-d
        \If{\Call{Length}{mode} == 4}
        \State $alpha$ = $data[3]$
        \State \Return \Call{Concatenate}{$rgb$, $alpha$}
        \Else
        \State \Return $rgb$
        \EndIf
      \EndIf
      \State \Return \Call{$Function$}{$data$, $args$} \Comment not RGBA
      \EndFunction
    \end{algorithmic}
  \end{algorithm}

  \newpage

  \begin{algorithm}
    \caption{Gauss Noise Generator}
    \label{GaussGenerator}
    \begin{algorithmic}[1]
      \Function{Gauss}{$\mu$, $\sigma$}
      \State $cache$ = $!cache$ \Comment {$cache$ should be static}
      \If{$cache$}
      \State \Return $\mu + \sigma * z1$ \Comment {Use the already caculated number.}
      \EndIf
      \\
      \State $u, v$ = two random number having uniform distribution in [0,1)
      \State $z0$ = $\sqrt{-2ln(u)}cos(2\pi v)$
      \State $z1$ = $\sqrt{-2ln(u)}sin(2\pi v)$
      \State \Return $\mu + \sigma * z0$
      \EndFunction
    \end{algorithmic}
  \end{algorithm}

  \begin{algorithm}
    \caption{Salt And Pepper Noise Generator}
    \label{SaltPepperGenerator}
    \begin{algorithmic}[1]
      \Function{SaltAndPepper}{$level$ = 256, $ps$ = 0.0, $pp$ = 0.0, $origin$}
      \LineComment {$ps$} the probability of salt
      \LineComment {$pp$} the probablity of pepper
      \LineComment {$origin$} the origin value
      \State $rs$ = a random number between [0, 1)
      \If{$rs$ $\textless$ $ps$}
      \State \Return $level$ - 1
      \ElsIf{$rs$ $\textgreater$ (1 - $pp$)}
      \State \Return 0
      \Else
      \State \Return origin
      \EndIf
      \EndFunction
    \end{algorithmic}
  \end{algorithm}

  \newpage

  \begin{algorithm}
    \caption{Arithmetic Mean Filter}
    \label{Arithmetic}
    \begin{algorithmic}[1]
    \Function{Arithmetic}{$data$, $size$}
    \State $m,n$ = size
    \State $filter$ = an mxn matrix with elements equals to $\frac{1}{m\times n}$
    \State \Return \Call{filter2d}{$data$, $filter$}
    \EndFunction
    \end{algorithmic}
  \end{algorithm}

  \begin{algorithm}
    \caption{Geometric Mean Filter}
    \label{Geometric}
    \begin{algorithmic}[1]
    \Function{Geometric}{$data$, $size$}
    \State $M, N$ = $data$'s height, width
    \State $m, n$ = size
    \For{each $(x,y)$ in $data$}
    \State $neighbors$ = get $m\times n$ neighbors of $(x,y)$
    \State $neighbors^{\frac{1}{m\times n}}$ element-wise
    \State $r$ = product of elements in $neighbors$
    \State put $r$ to $(x,y)$ in $\mathbf{result}$
    \EndFor
    \State \Return $result$
    \EndFunction
    \end{algorithmic}
  \end{algorithm}

  \newpage

  \begin{algorithm}
    \caption{Harmonic Filter}
    \label{Harmonic}
    \begin{algorithmic}[1]
    \Function{Harmonic}{$data$, $size$}
    \State $reciprocal$ = get reciprocal of $data$ element-wise
    \State $denominator$ = \Call{Arithmetic}{$reciprocal$, $size$}
    \State \Return the reciprocal of $denominator$ element-wise
    \EndFunction
    \end{algorithmic}
  \end{algorithm}

  \begin{algorithm}
    \caption{Contraharmonic Filter}
    \label{Contraharmonic}
    \begin{algorithmic}[1]
    \Function{Contraharmonic}{$data$, $size$, $q$}
    \State $m,n$ = $size$
    \State $numerator$ = $data^{q+1}$ element-wise
    \State $denominator$ = $data^{q}$ element-wise
    \State $filter$ = an mxn matrix with elements equals to $1$
    \State \Return $\frac{\Call{Arithmetic}{numerator}}{\Call{Arithmetic}{denominator}}$ element-wise
    \EndFunction
    \end{algorithmic}
  \end{algorithm}

  \begin{algorithm}
    \caption{Statistic Filter}
    \label{Statistic}
    \begin{algorithmic}[1]
    \Function{Statistic}{$data$, $size$, $percent$}
    \LineComment $percent$ = 0, min filter
    \LineComment $percent$ = 50, median filter
    \LineComment $percent$ = 100, max filter
    \For{each $(x,y)$ in $data$}
    \State $neighbors$ = get neighbors of $(x,y)$ within $size$ if exists
    \State sort $neighbors$
    \State $index$ = $\frac{percent}{100}\times($\Call{LENGTH}{$neighbors$}$-1)$
    \State $i\_below, i\_above$ = \Call{Floor}{index}, \Call{Ceil}{index}
    \State $r$ = $(neighbors[i\_below] + neighbors[i\_above]) / 2$
    \State put $r$ to $(x,y)$ in $\mathbf{result}$
    \EndFor
    \State \Return $result$
    \EndFunction
    \end{algorithmic}
  \end{algorithm}

\end{document}